
\subsection{Evaluation metrics}
\vspace*{-0.1cm}
We are interested in analysing speaker identification system performance, and particularly, the influence of some characteristics, in the training and the testing data. Thus, one speaker in a given show is considered as the unit of analysis, the so-called $SpkShow$. One speaker appearing in 2 different videos is considered as 2 distinct $SpkShow$. 

In this analysis, we adopt the point of view of the references: for each $SpkShow_i$ in the reference is computed a performance measure of the biometric system, defined as the F-measure of the detection of $SpkShow_i$. More precisely, defining $T^{ref}_i$ the total duration of $SpkShow_i$ in the reference, $T^{test}_i$ the total duration where $SpkShow_i$ is the system response and $T^{corr}_i$ the total duration of correct identification of $SpkShow_i$, Precision and Recall can be computed for each $SpkShow_i$:
\begin{itemize}
\item $Precision_i=\frac{T^{corr}_i}{T^{test}_i}$ ~~~~~~$Recall_i=\frac{T^{corr}_i}{T^{ref}_i}$
\item $Fm_i=\frac{2*Precision_i*Recall_i}{Precision_i+Recall_i}$
\end{itemize} 

Thus, $Fm_i=0$ means that $SpkShow_i$ was never correctly identified, whereas $Fm_i=1$ means that $SpkShow_i$ is perfectly identified, without miss detection nor false alarm.

\subsection{REPERE Corpus~\cite{Giraudel2012}}

The REPERE challenge~\cite{KAHN--CBMI--2012} is an evaluation campaign on multimodal person recognition (phase 1 took place in January 2013 and phase 2 in January 2014).  The systems evaluated in our experiments are the "mono-modal" systems (based only on voice-based speaker identification) submitted to phase2, on data set named \texttt{test2}, composed of 62 videos recorded from 8 different types of show (including news and talk shows) broadcasted from two French TV channels. 10 hours of speech are annotated, and contain 477 non-anonymous $SpkShow$, which have on average 6.2 speech turns each, for a mean duration of speech turn equal to 12.1s


%In table \ref{tab:test2} we detail the speaker repartition on the \texttt{test2} corpus.


%begin{table}[ht]
 % \centering
  %\begin{tabular}{|c|c|}
    %\hline
    %Part                                & \texttt{test2}    \\
    %\hline    
    %\# videos                           & 62                \\         
    %Annotated duration                  & 10 h.             \\
    %\hline      
    %$SpkShow$                           & 477               \\
    %\# speaker turns                    & 2981              \\
    %speech duration                     & 35929             \\
    %mean duration speech turns in s.    & 12.1 (15.32)      \\
    %\# speech turn per $SpkShow$        & 6.2 (10.7)        \\ 
   % \hline                              
  %\end{tabular}
  %\caption{Corpus size for the annotated part, number of $SpkShow$, speakers turns and speech duration for non-anonymous speakers. Mean duration of speech turns and number of speech turns par $SpkShow$, in parenthesis is the stantard deviation}
  %\label{tab:test2}  
%\end{table}


\subsection{Systems description}
\label{sec:systems}
brief description of the systems: training data, modelling, decision...
\subsection{PERCOL}

\subsection{QCOMPERE}

\subsection{SODA}





