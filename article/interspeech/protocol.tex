



In this paper, we are interested in analyzing the performances obtained by supervised speaker identification system, and particulary in their capability to identify speakers, according to their characteristics, for instance in terms of speech turns etc. As the speech turns properties depend on the show in which the speaker appears, one speaker in one show is considered as the unit of analysis, the so-called $SpkShow$ (see table~\ref{notation} for notation used in the rest of the paper). One speaker appearing in 2 different videos is considered as 2 distinct $SpkShow$.

\begin{table}[ht]
  \centering
    \begin{tabular}{|c|l|}
        \hline 
        Notation & Description \\ 
        \hline 
        \hline
        $SpkShow$ & All speech segments of a speaker in a video \\ 
        %$SpkSeg$  & A speaker turn \\
        $T^{ref}_i$ & the total duration of the speech\\
        & of  $SpkShow_i$, in the reference \\
        $T^{test}_i$ & the total duration where $SpkShow_i$\\
        & is recognized by the automatic system\\
        $T^{corr}_i$ & the total duration of correct\\
        & identification of $SpkShow_i$\\
        \hline 
    \end{tabular} 
    \caption{Notation used in the rest of the paper}
    \label{notation}
\end{table}

In this analysis, we adopt the point of view of the references: for each $SpkShow_i$ in the reference is computed a performance measure of the biometric system, defined as the F-measure of the detection of $SpkShow_i$. More precisely, considering the definition given in \ref{notation}, Precision and Recall can be computed for each $SpkShow_i$:
\begin{itemize}
\item $Precision_i=\frac{T^{corr}_i}{T^{test}_i}$
\item $Recall_i=\frac{T^{corr}_i}{T^{ref}_i}$
\item $Fm_i=\frac{2*Precision_i*Recall_i}{Precision_i+Recall_i}$
\end{itemize} 

Thus, $Fm_i=0$ means that $SpkShow_i$ was never correctly identified, whereas $Fm_i=1$ means that $SpkShow_i$ is perfectly identified, without miss detection nor false alarm.

\subsection{REPERE Corpus~\cite{Giraudel2012}}

The REPERE challenge~\cite{KAHN--CBMI--2012} is an evaluation campaign on multimodal person recognition (phase 1 took place in January 2013 and phase 2 in January 2014). The main objective is to answer the two following questions at any instant of the video: \emph{``who is speaking?"} \emph{``who is seen?"}. All modalities available (audio, image, external data ...) can be used for answering these questions. 

The subset used in our experiments correspond to the \texttt{test2}. It is composed of videos recorded from eight different shows (including news and talk shows) broadcasted from two French TV channels. In table \ref{tab:test2} we detail the speaker repartition on the \texttt{test2} corpus.


\begin{table}[ht]
  \centering
  \begin{tabular}{|c|c|}
    \hline
    Part                                & \texttt{test2}    \\
    \hline    
    \# videos                           & 62                \\         
    Annotated duration                  & 10 h.             \\
    \hline      
    $SpkShow$                           & 477               \\
    \# speaker turns                    & 2981              \\
    speech duration                     & 35929             \\
    mean duration speech turns in s.    & 12.1 (15.32)      \\
    \# speech turn per $SpkShow$        & 6.2 (10.7)        \\ 
    \hline                              
  \end{tabular}
  \caption{Corpus size for the annotated part, number of $SpkShow$, speakers turns and speech duration for non-anonymous speakers. Mean duration of speech turns and number of speech turns par $SpkShow$, in parenthesis is the stantard deviation}
  \label{tab:test2}  
\end{table}


\subsection{Systems description}
\label{sec:systems}
brief description of the systems: training data, modelling, decision...

\begin{table*}[t]
  \centering
  \begin{tabular}{|c|c|c|c|}
    \hline
    System          & PERCOL                                        & QCOMPERE                                        & SODA      \\
    \hline    
    Diarization     & i-vector \cite{charlet2013}                   &  multi-stage speaker diarization \cite{Barras2006} &  i-vector \cite{dupuy2014}         \\
                    & + overlapping speech detection                &                                                 &           \\
    \hline    
    SID             & ALIZE v3.0 toolkit \cite{larcher2013}         & Bob toolkit \cite{bob2012}                      &  ALIZE v3.0 toolkit \cite{larcher2013}  \\
    \hline    
    feature         & 19 LFCC + $\delta$ coef,                      & 15 PLP-like cepstrum coef~\cite{Hermansky1990}  &  19 MFCC + $\delta$ coef,         \\
                    & + $\delta$ energy + 11 $\delta$$\delta$ coef  & 15 $\delta$ coef + $\delta$ energy              &  + $\delta$ energy + 11 $\delta$$\delta$ coef  \\
    \hline    
    UBM             & gender independent                            & 256 diagonal Gaussians                          &  gender independent         \\
                    & 512 GMM                                       &                                                 &  1024 GMM         \\
    \hline    
    i-vector        & 200 dim TVS estimated from                    & 400 dim TVS trained using $39356$               &  300 dim TVS estimated from         \\
                    & 1200 spk and 7500 sessions                    & speech segments (around $15$ seg/spk)           &  680 spk and 4150 sessions         \\
    \hline    
    Normalisation   & cepstral mean subtraction                     & Feature warping normalization with a            &  cepstral mean subtraction         \\
                    & and variance normalization                    & sliding window of $3$~s.~\cite{Pelecanos2001}.  &   and variance normalization        \\
    \hline    
    Training data   & min 30s, max 2mn30 (if higher                 & 1 \emph{i-vector}/spk, min 30s                  &  680 speakers, min 1mn , max 12 min         \\
    for i-vector    & a set of i-vectors is extracted)              & REPERE+ETAPE+French radio                       &  REPERE+ETAPE+French radio + web         \\
                    & REPERE corpus                                 & $706$~spk \emph{identity} vertices              &           \\            
    \hline    
    Decision        & CDS joined with Within-Class                  & PLDA                                            &  PLDA         \\
                    & covariance normalization for                  &                                                 &  length-normalization and whitening         \\
                    & session/channel compensation                  &                                                 &           \\
    \hline                              
  \end{tabular}
  \caption{System comparison, TVS : total variability space, CDS: Cosine Distance Scoring}
  \label{tab:system}  
\end{table*}

\subsubsection{PERCOL}
The monomodal speaker ID system of PERCOL consortium relies on a typical i-vector framework. It is applied on each cluster provided by a speaker diarization system, which provides speech segments belonging to speakers as well as speaker clusters.
The diarization system is the one presented in \cite{charlet2013}. It is a sequential processing using firstly Bayesian Information Criterion and then Cross-likelihood Criterion. A special attention is paid for overlapped speech for TV-debates, where the amount of overlapped speech is significant. For these shows uniquely, overlapped speech segments are first detected and discarded from the clustering process, and then reassigned to the 2 nearest speakers, in terms of temporal distance between speech segments.

The i-vector-based speaker ID system relies on the ALIZE v3.0 toolkit \cite{larcher2013}. 19 LFCC augmented with their delta coefficients, the delta energy, and 11 double delta coefficients are used for the feature extraction. Features are then normalized, file by file, by applying a cepstral mean subtraction and variance normalization. The i-vector extraction relies on a 200 dimension total variability space estimated from about 1200 speakers and 7500 sessions. The Universal Background Model (UBM) is gender independent, represented by a 512 component Gaussian Mixture Model. It is learnt on about 200h of speech.

In the REPERE context, training data for a speaker may vary from a very few seconds to more than 2 hours. For this reason and based on the experience gained from the NIST evaluation campaigns, the extraction of the training i-vectors is carried out according to the following couple of rules : (1) for a given speaker, an i-vector is extracted only if a minimum 30s long training data are available, (2) if training data for a given speaker is longer than 2mn30, a set of i-vectors is extracted, each of them on the basis of 2mn30 duration, the last one having to respect the rule (1).
As mentioned above, a testing i-vector is extracted from all the speech segments gathered in a same cluster by the speaker diarization system. This cluster-based i-vector permits to handle overlap speech since a segment may be processed twice if it is attributed to different clusters during the speaker diarization process.

For the decision, the Cosine Distance Scoring (CDS) joined  with Within-Class covariance normalization for session/channel compensation is used to compare a couple of training and testing i-vectors. Matrix involved in the WCCN normalization is estimated on 571 speakers and 4384 sessions. When multiple training i-vectors are available to model a speaker, the maximum score is involved in the speaker ID decision process. Finally, all the scores are t-normed, notably for the open-set speaker identification task not discussed in this paper.




\subsubsection{QCOMPERE}
Acoustic feature vectors are extracted from the speech signal on the 0-8kHz
bandwidth every $10$ms using a $30$ms Hamming window.
They consist of 15 PLP-like cepstrum coefficients~\cite{Hermansky1990}
with 15 delta coefficients and delta energy, for a total of 31 features.
Feature warping normalization is performed using a sliding window of $3$~seconds
in order to reduce the effect of the acoustic environment~\cite{Pelecanos2001}.
The Universal Background Model is a mixture
of 256 diagonal Gaussians trained on a multilingual broadcast corpus.
Then, three annotated data sources were used to train one \emph{i-vector} per speaker: the REPERE training~\cite{Giraudel2012}, the ETAPE training and development data~\cite{Gravier2012} and additional French politicians data extracted from French radio broadcast.
Only speakers with more than $30$ seconds training data were kept,
resulting in $706$~speaker \emph{identity} vertices.
The \emph{total variability space} is trained using $39356$ speech segments
of variable length (few seconds to several minutes) collected
over the target speakers (around $15$ segments/speaker).
$400$-dimensional \emph{i-vector} is considered for characterization of the speech segment.
In test phase, only speaker factor (channel factor is kept fixed equal
to the dimension of \emph{i-vector} i.e. $400$) of PLDA is varied to find
the optimal performance of the speaker identification system
on development data set.
Before PLDA, \emph{i-vectors} are length-normalized by two iterations
of Eigen Factor Radial algorithm \cite{Bousquet2011}.


\subsubsection{SODA}
The speaker identification system is based on a i-vector/PLDA framework developed using the ALIZE v3.0 toolkit \cite{larcher2013}. The acoustic frame composed of 50 coefficients (19 MFCC, 19 delta and 11 delta-delta and the delta energy) is extracted every 10 ms in a windows of 25.6 ms. A energy-based frame selected is applied before to subtract the cepstral mean and normalize the variance. 

The same training corpus is used to train the UBM, the total variability matrix, the PLDA matrices and the target i-vectors. This training corpus is composed of 680 speakers corresponding to 4150 sessions for a total of 103 hours of speech. The UBM is composed of 1024-distribution with diagonal co-variance matrix. The total variability space is set to rank 300 and the low rank speaker and channel matrices from the PLDA model are set to 150 and 40 respectively. After i-vector extraction, length-normalization and whitening is perfomed.

During the test phase, the signal is automatically segmented and clustered \cite{dupuy2014}. A i-vector is extracted and normalized for each cluster. A close-set speaker identification test is performed to assign the identity of one of the 680 target i-vectors to the cluster. 

 



