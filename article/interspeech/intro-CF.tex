For about five years, tremendous progress has been made in the speaker recognition field, especially for the speaker detection task (i.e. to verify whether a claimed
speaker is speaking during a given segment of speech) applied in a phone environment. Supported by the evaluation campaigns organized by the National Institute of Standards and Technology (NIST)\cite{greenberg2013,greenberg2014}, this progress mainly relies on the development of the i-vector paradigm, which has definitely overtaken classical UBM/GMM \cite{bimbot2004}, or SVM \cite{wan2000} approaches. Inspired from the Joint Factor Analysis (JFA), which had already been applied with success in speaker detection, and which aims at estimating speaker and channel/session subspaces separately, the simpler and powerful i-vector-based modeling paradigm \cite{dehak2011} makes no distinction between both of subspaces thanks to a single total variability space, which covers both the speaker and session/channel variability. In addition to improving performance of the speaker detection systems, the i-vector-based paradigm takes advantage of the low dimension spaces involved, in terms of complexity reduction, compared with the JFA framework. A large amount of studies has been dedicated to the enhancement of this paradigm, still in speaker detection, by coupling it with different channel compensation techniques\cite{dehak2011,bousquet2012,kanagasundaram2014}, or by investigating various scoring approaches, which directly embeds channel compensation\cite{kenny2010,dehak2011,garcia2011,jiang2012,bousquet2014}. Faced to this success, the i-vector paradigm has been introduced in other speech processing fields like speaker diarization \cite{dupuy2012,shum2013,senoussaoui2014}, or language recognition \cite{martinez2011,dehak2011b}. \\

As mentioned above, the recent contributions in speaker recognition have been devoted to phone conversation recordings for the specific task of speaker detection.  A very few recent studies have concerned speaker identification in TV broadcast based on state-of-the-art speaker recognition systems. However, this context implies some non-trivial particularities such as the widely varying amount of training data per speaker according to their role (recurrent anchor speakers, popular or punctual speakers, etc.), the properties of speech segments - duration, number, acoustic quality, etc - implied in the identification decision while processing an entire TV show, generally issued from a preliminary speaker diarization step, and finally the coverage of speaker dictionary used by the system and its direct impact on an open-set identification task (closer to real life applications).








