\documentclass[a4paper]{article}
\usepackage{INTERSPEECH2014,amssymb,amsmath,epsfig}
\setcounter{page}{1}
\sloppy     % better line breaks
\ninept
%SM below a registered trademark definition
\def\reg{{\rm\ooalign{\hfil
     \raise.07ex\hbox{\scriptsize R}\hfil\crcr\mathhexbox20D}}}

%% \newcommand{\reg}{\textsuperscript{\textcircled{\textsc r}}}
\title{Speakers identification in broadcast TV: facilities and barriers}

%%%%%%%%%%%%%%%%%%%%%%%%%%%%%%%%%%%%%%%%%%%%%%%%%%%%%%%%%%%%%%%%%%%%%%%%%%
%% If multiple authors, uncomment and edit the lines shown below.       %%
%% Note that each line must be emphasized {\em } by itself.             %%
%% (by Stephen Martucci, author of spconf.sty).                         %%
%%%%%%%%%%%%%%%%%%%%%%%%%%%%%%%%%%%%%%%%%%%%%%%%%%%%%%%%%%%%%%%%%%%%%%%%%%
%\makeatletter
%\def\name#1{\gdef\@name{#1\\}}
%\makeatother
%\name{{\em Firstname1 Lastname1, Firstname2 Lastname2, Firstname3 Lastname3,}\\
%      {\em Firstname4 Lastname4, Firstname5 Lastname5, Firstname6 Lastname6,
%      Firstname7 Lastname7}}
%%%%%%%%%%%%%%% End of required multiple authors changes %%%%%%%%%%%%%%%%%

\makeatletter
\def\name#1{\gdef\@name{#1\\}}
\makeatother \name{{\em Author Name$^1$, Co-author Name$^2$}}

\address{$^1$Author Affiliation \\
  $^2$Co-author Affiliation \\
{\small \tt author@university.edu, coauthor@company.com}}

%\twoauthors{Karen Sp\"{a}rck Jones.}{Department of Speech and Hearing \\
%  Brittania University, Ambridge, Voiceland \\
%  {\small \tt Karen@sh.brittania.edu} }
%  {Rose Tyler}{Department of Linguistics \\
%  University of Speechcity, Speechland \\
%  {\small \tt RTyler@ling.speech.edu} }

%
\begin{document}
\maketitle
%
\begin{abstract}
...
\end{abstract}
\noindent{\bf Index Terms}: speaker recognition, error analysis


\section{Introduction}

\section{\label{notation}Notation}

\begin{tabular}{|c|l|}
\hline 
Notation & Description \\ 
\hline 
\hline
$SpkShow$ & All speech segments of a speaker in a video \\ 
$SpkSeg$  & A speaker turn \\
$T^{ref}_i$ & the total duration of the speech\\
& of  $SpkShow_i$, in the reference \\
$T^{test}_i$ & the total duration where $SpkShow_i$\\
& is recognized by the automatic system\\
$T^{corr}_i$ & the total duration of correct\\
& identification of $SpkShow_i$\\

\hline 
\end{tabular} 

 
\section{Systems description}
brief description of the systems: training data, modelling, decision...
\subsection{PERCOL}

\subsection{QCOMPERE}

\subsection{SODA}


\section{Performance analysis}
\input{spkshow_analysis.tex}

\section{Performance Prediction}

In this section, the aim is to predict the performance of the speaker, from his characteristics in terms of training data and speech turns properties. If we are able to predict, reliably, if the $Spkshow$ will be correctly recognized or not, when analysing on the main features contributing to this prediction, we can identify what are the features that  facilitates or hamper the identification, for a given system.

At each $SpkShow$ is associated the maximal $Fm$ obtained accross systems. Doing so, we do not focus on a particular system, but we try to explain "the-best-we-can-do" performance for each $SpkShow$.

\subsection{Detection}


\subsection{Prediction}

\section{Cross-show extension}





\section{Conclusions}



\section{Acknowledgements}


\newpage

\eightpt
\bibliographystyle{IEEEtran}

\begin{thebibliography}{10}

\bibitem[1]{ES1} ....

\end{thebibliography}
\end{document}
